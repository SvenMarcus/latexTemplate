\chapter{Einführung in VirtualFluids}

\textit{VirtualFluids} ist eine Anwendung zur Lösung der Transportgleichungen von Fluidströmungen mithilfe der \textit{Lattice-Boltzmann-Methode}, die am \textit{Institut für rechnergestützte Modellierung im Bauingenieurwesen} an der TU Braunschweig entwickelt wird \cite{Kutscher2020}. Die Software ist für massiv-parallele Anwendungen optimiert und kann sowohl auf herkömmlichen Rechenkernen (\textit{CPU}) als auch auf Grafikkarten (\textit{GPU}) Berechnungen ausführen. \textit{VirtualFluids} stellt außerdem Funktionalitäten für die typischen drei Bestandteile von Simulationen, \textit{Pre-Processing}, Berechnung und \textit{Post-Processing}, bereit. Im Rahmen des \textit{Pre-Processings} ist die Software in der Lage einfache geometrische Strukturen zu generieren oder komplexere Daten einzulesen, die anschließend auf ein diskretisiertes Berechnungsgitter übertragen werden. Bei der Lösung von Transportproblemen verwendet \textit{VirtualFluids} verschiedene Strömungs-Kernel deren Berechnungen sowohl über \textit{Threads} als auch über das \textit{Message Passing Interface} (\textit{MPI}) parallelisiert werden können \cite{MPIForum1994}. Das \textit{Post-Processing} wird unter anderem durch Aufzeichnung der berechneten Größen in visualisierbaren Dateiformaten unterstützt.