\chapter{Implementierung einer Python-\newline Schnittstelle für VirtualFluids}

Die Konfiguration von Simulationen in \textit{VirtualFluids} erfolgt in der \textit{main}-Methode der \textit{C++}-Anwendung. Die Applikation muss aufgrund der Änderung des Quelltextes
für jeden Simulationsfall neu kompiliert werden. \textit{VirtualFluids} kann daher nicht als kompiliertes Artefakt an Endnutzer ausgeliefert werden.
Des Weiteren existieren keine Abstraktionen zur Vereinfachung der Konfiguration. Ein Anwender muss daher nicht nur mit dem fachlichen Hintergrundwissen, 
sondern auch mit der Struktur der Applikation vertraut sein, um eine Simulation korrekt durchzuführen.\\

\textbf{TODO: DSL Grundlagen}

\section{Generalisierung der Konfiguration}


\section{Implementierung der Python-Schnittstelle}\cite{Fowler2011}